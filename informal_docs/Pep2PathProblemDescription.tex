\documentclass{article}

\title{Pep2Path Year 4 Project}

\begin{document}

\section*{Problem Description}

An important field of research for biologists is the identification of new biological products with novel properties.
However with the addition of each component to a chain of biological molecules, the chemical space grows exponentially.
(For example considering the relatively limited case of the twenty 'standard' AAs and a peptide of length 4, there are
\(20^4\) possible combinations... Around 160,000!) \\

An exhaustive search would be impossible, so instead we identify biological products produced in the wild.
For an organism to spend biological resources producing some product, we can assume it confers some evolutionary advantage.
For example, a strain of bacteria might produce a biological product that kills another competitor species of bacteria
(which we can then use to ourselves kill a potentially deadly strain of bacteria).
In this way we can use the natural algorithm of evolution as a heuristic to explore the chemical space for us and identify potentially
interesting biological products. \\

We can then try to identify these potentially interesting biological products by, for example, analysis of chemical structure
for similarity to other interesting products, or by observing an interesting behaviour across several species which
produce a particular product. However we also want to isolate the particular biosynthetic gene cluster (BGC) which produces
this novel product - this is a generalisation of peptidogenomics, the matching of peptides to geonomes. 
By observing species with the same BGC produced this product (clustering), we can generally assume this BGC has a good chance of 
producing the product in question. \\

Pep2Path is a software that can probabilistically match peptidic products (a subset of our field of interest) to BGCs. 
Given the output from a mass spectrometer we can reconstruct composition of a peptide and order of amino acids at consecutive mass-peaks, 
using a mass-translation table, and given a geonome sequence, we can use tools like antiSMASH and NRPSPredictor to 
identify potential candidate BGCs in the geonome. Pep2Path then takes these two outputs and calculates the most likely matches of
sequence to BGC. \\

However, Pep2Path is not very well-maintained, and its codebase has proven to be largely unmaintainable and inextensible.
We wish to implement a new software which can perform some of the useful functionality of the original Pep2Path,
while offering more useful features and the potential for future development. 

\end{document}